% Inbuilt themes in beamer
\documentclass{beamer}

% Theme choice:
\usetheme{CambridgeUS}

% Title page details: 
\title{Assignment $4$\\ Probability and random Variables} 
\institute{Indian Institute of Technology Hyderabad}
\author{Shreyas Wankhede}
\date{\today}
\logo{\large \LaTeX{}}


\begin{document}

% Title page frame
\begin{frame}
\titlepage 
\end{frame}

% Remove logo from the next slides
\logo{}


% Outline frame
\begin{frame}{Outline}
    \tableofcontents
\end{frame}


% Lists frame

\section{figure}
\begin{frame}{figure}
\begin{figure}
  \includegraphics[width=3in,height=2.4in]{Figure_1.png}
  \caption{fig-1}
  \label{fiure-1}
\end{figure}
\end{frame}
\section{Example}
\begin{frame}{Example}

\textbf{Papoulis book example 2.19}\\\vspace{5mm}
The below example is an illustration that emphasizes that three events can be independent in pairs but not independent\\\vspace{3mm}
Suppose that the events A,B,C of Fig-1 have the same probability.\\
\begin{align}
    P(A)=P(B)=P(C)=\dfrac{1}{5}\nonumber
\end{align}
and the intersections AB,AC,BC,ABC also have the same probability
\begin{align}
    p=P(AB)=P(BC)=P(AC)\nonumber
\end{align}



\end{frame}

\section{Independence of events}
\begin{frame}{Independence of events}
 \textbf{Independence of three events:}\\
 The events $A_1, A_2 and A_3$ are called (mutually) independent if they are independent in pairs:\\
 \begin{align}
     P(A_iA_j)=P(A_i)\times P(A_j)\label{eq:1}
 \end{align}
 and
 \begin{align}
     P(A_1A_2A_3)=P(A_1)\times P(A_2)\times P(A_3)\label{eq:2}
 \end{align}
\end{frame}
% Blocks frame
\section{statements}
\begin{frame}{statements}
   \begin{enumerate}[label=\Alph*]
       \item 
       if p=\dfrac{1}{25}\\\vspace{2mm} then these events are independent in pairs but they are not independent because
    \begin{align}
          P(ABC)\neq P(A)\times P(B)\times P(C)\nonumber
    \end{align}
    
      \item
      if p=\dfrac{1}{125}\\\vspace{2mm} then 
      \begin{align}
          P(ABC)=P(A)\times P(B)\times P(C)\nonumber
      \end{align}
      but the events are not independent because 
      \begin{align}
          P(AB)\neq P(A)\times P(B)\nonumber
      \end{align}
    
   \end{enumerate}

\end{frame} 

\section{explanation}
\begin{frame}{Explanation}
 From the independence of the events A, B, and C it follows that: \\
 \begin{enumerate}
     \item 
     Anyone of them is independent of the intersection of the other two.
     Indeed, from \eqref{eq:1} and \eqref{eq:2} it follows that 
     \begin{align}
         P(A_1A_2A_3) = P(A_1)\times P(A_2)\times P(A_3) = P(A_1)\times P(A_2A_3)\label{eq:3}
     \end{align}
     Hence the events $A_1$ and $A_2 A_3$ are independent
     
     \item
     If we replace one or more of these events with their complements, the resulting events are also independent.Indeed since,
     \begin{align}
         A_1A_2= A_1A_2A_3 \cup A_1A_2\overline A_3; \hspace{3mm}P(\overline A_3)=1 -P(A_3)\nonumber
     \end{align}
     we conclude with \eqref{eq:3} that 
     \begin{align}
         P(A_1A_2\overline A_3)=P(A_1A_2) -P(A_1A_2)\times P(A_3)=P(A_1)\times P(A_2)\times P(\overline A_3))\nonumber
     \end{align}
     
     \item
      Anyone of them is independent of the union of the other two. 


    \end{enumerate}
 

\end{frame}

\begin{frame}[t]{}
 To show that the events $A_1$ and $A_2\cup A_3$ are independent, it suffices to show that the events $A_1$ and $\overline{A_2\cup A_3}=\overline A_2\overline A_3$ are independent. This follows from 1 and 2


\end{frame}

\section{Generalization}
\begin{frame}{Generalization}
    Generalization: The independence of n events can be defined inductively:\\ Suppose that 
    we have defined independence of k events for every $k < n$. We then say that the events $A_1, ... , A_n$ are independent if any $k < n$ of them are independent and 
    \begin{align}
        P(A_1....A_n)=P(A_1)....P(A_n)
    \end{align}
    This completes the definition for any n because we have already defined independence for $n=2$

\end{frame}


\end{document}
